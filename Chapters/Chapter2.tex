% Chapter 2

\chapter{Scope} % Main chapter title
\label{Chapter2} % For referencing the chapter elsewhere, use \ref{Chapter2} 

The main work of this research is to study the offline analysis of the ScaRR's
model~\cite{toffaliniScaRRScalableRuntime2019} over a set of programs written in
C/C++. Specifically, we write a tool to extract offline measurement using the
ScaRR control-flow model, verify the scalability of the algorithm and study its
limitation.

To sum up, the main contributions of this thesis are:
\begin{enumerate}
	\item Propose an implementation of ScaRR's offline measurement extractor.
	\item Verify the scalability of ScaRR's offline measurement algorithm.
	\item Study the algorithm limitation.
\end{enumerate}

\section{Outline}
\label{sec:outline}

\vspace{0.5cm}
\noindent \textbf{Chapter~\ref{Chapter3}:} We discuss the background. We present
control-flow attack and remote attestation to attest control-flow attack. We
also discuss ScaRR control-flow model. Last, we present LLVM, which we use to
extract control-flow graph (CFG); to mark the basic block as checkpoints and
find the list of actions between those checkpoints.

\vspace{0.5cm}
\noindent \textbf{Chapter~\ref{Chapter4}:} We present the methodology based on
the scope of the contributions. We start by discussing the threat model. We then
mainly shows the algorithms and implementation of the offline measurement
extractor.

\vspace{0.5cm}
\noindent \textbf{Chapter~\ref{Chapter5}:} We show the result of the
implementation and experiment. We demonstrate how to run the LLVM pass. Finally,
we present the control flow result for some programs that we test. We also
discuss the complexity analysis of the implementation. Last, we present one of
the programs as a case study. We show the breakdown of the process from showing
the code, get the control-flow graph, and ultimately how we get the offline
measurement result.

\vspace{0.5cm}
\noindent \textbf{Chapter~\ref{Chapter6}:} We discuss the result, limitation and
future works for the research.

\vspace{0.5cm}
\noindent \textbf{Chapter~\ref{Chapter7}:} We close this thesis with conclusion
that we gather after the research.