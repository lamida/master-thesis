% Chapter 5

\chapter{Results} % Main chapter title

\label{Chapter5} % For referencing the chapter elsewhere, use \ref{Chapter5} 

In this research we used the offline measurement generator in getting the 
measurement across different real world programs. We calculate the ScaRR 
control flow information for each of the programs and we analyse and present the result in this chapter. 
The analysis and source code of the program is available in this Github repository: \url{https://github.com/lamida/scarr-sample-program/}.

In this research we choose 4 large open source projects for analysis. We download the source code and checkout the latest version of the code. 
The first software is Redis $6.2.4$  \cite{RedisRedis2021}, a what so called data structure server that is widely used in real world. 
The Redis source build consist of the server binary and the client cli. We analyse both of the program.

The second program we analyse is bzip2 $1.08$ \cite{Bzip2Bzip2}. Bzip2 is free and open source file compression program. The third program is openssl $1.1.1j$ \cite{OpensslOpenssl2021}. OpenSSL is full-featured toolkit for TLS protocol. 

The last suite program we analyse is coreutils $8.32$ \cite{CoreutilsCoreutils2021}. Coreutils is suite of Unix utilities for file, shell and text manipulation.

For each of program, we are collecting the following measurements:

\begin{itemize}
    \item source code lines
    \item IR lines
    \item number of basic blocks (nBB)
    \item number of ScaRR measurements (nM)
    \item number of checkpoints (nCP)
    \item number of LoA (nLoA)
\end{itemize}

\section{ScaRR Control Flow Result}

\xt{Elaborate the results, add charts for better visualization than just table.}

\csvautolongtable{csv/coreutils.csv}
\xt{Find a way to add caption to this long table}

\begin{table}[hbtp]
    \centering
    \csvautotabular{csv/redis.csv}
    \caption{Redis ScaRR measurements}
\end{table}

\begin{table}[hbtp]
    \centering
    \csvautotabular{csv/misc.csv}
    \caption{Some additional programs ScaRR measurements}
\end{table}

\section{Complexity Analysis}

TBD

\section{Case Study}

TBD