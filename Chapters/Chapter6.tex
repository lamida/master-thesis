% Chapter 5

\chapter{Discussion and Future Works} % Main chapter title

\label{Chapter6} % For referencing the chapter elsewhere, use \ref{Chapter6} 

In Chapter~\ref{Chapter5}, we present the result of the implementation of
ScaRR's offline measurement generator. We learn that the algorithm time
complexity is linear to the input. The input here is the size of the program. We
measure the size of the program as a number of lines. The number of lines is
also linear with the size of intermediate representation and the number of basic
blocks. Ultimately, the most important finding we confirm that the number of
measurement is also linear with the size of the program. With that confidence
that we can use the algorithm to do runtime remote attestation for a complex
program.

In this study, we identify the limitation of the algorithm. We can use this for
possible future works. First, we do not test the algorithm with a program that
contains recursion, signals, and exception. We also just test the algorithm with
single-threaded program. We do not run the algorithm on a program that is
written by a programming language other than C. 

We also noticed the limitation on the algorithm cannot detect non-control data
attack which do not modify the program control-flow. The algorithm do know
whether a program is going to a loop. Unfortunately, the algorithm do not know
how many iteration a program should run. The offline measurement cannot know f
there is any attack that modifies the number of the loop. 