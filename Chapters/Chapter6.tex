% Chapter 5

\chapter{Discussion and Future Works} % Main chapter title

\label{Chapter6} % For referencing the chapter elsewhere, use \ref{Chapter6} 

In Chapter~\ref{Chapter5}, we discuss the result of ScaRR's offline measurement
generator. We learn that the algorithm time complexity is linear to the input.
The input here is the size of the program. We measure the size of the program as
the size of source code lines. The number of lines is also linear with the size
of intermediate representation and basic blocks number. The most important
finding we confirm the number of measurements is also linear with the program
size. We know now that we can use the algorithm to do runtime remote attestation
for a complex program.

In this study, we identify the limitation of the algorithm. We can use the
findings for possible future works. First, we do not test the algorithm with a
program that contains recursion, signals, and exception. We also only test the
algorithm with a single-thread program. We only run the analysis on C programs.

We also noticed the limitation on the algorithm that cannot detect non-control
data attacks. Non-control data attack does not modify program CFG. Next, the
algorithm does not track the number of loop iteration. The algorithm does know
whether a program is going to a loop. Unfortunately, the algorithm does not know
how many iterations a program should run. The offline measurement cannot know
when there is an attack that modifies the number of the loop. 